\section{Ce qui différencie un LAN d'un WAN}

\begin{frame}[fragile]
  \frametitle{Ce qui différencie un LAN d'un WAN}
\begin{center}
	\Huge{\bf\color{blue}Ce qui différencie un LAN d'un WAN}
\end{center}
\begin{flushright}
  \item L'existant
  \item La couche physique
  \item La couche liaison de données
  \item La couche réseau
\end{flushright}
\end{frame}

\begin{frame}[fragile]
  \frametitle{Ce qui différencie un LAN d'un WAN}
\textbf{Couche physique}
\begin{itemize}
	\item Le support de transmission doit pouvoir laisser passer
	\textbf{beaucoup} plus de données que dans un LAN
	\begin{itemize}
		\item multiplexage
	\end{itemize}
	\item L'existant et les « nouveaux » types de support
	\begin{itemize}
		\item le réseau téléphonique commuté (xDSL)
		\item le câble
		\item la fibre optique (SONET)
	\end{itemize}
	\item Les différents intervenants
\end{itemize}
\textbf{Couche liaison de données}
	\begin{itemize}
		\item Protocoles de liaison de données
	\end{itemize}
\textbf{Couche réseau}
	\begin{itemize}
		\item Routage des paquets
	\end{itemize}
\end{frame}

