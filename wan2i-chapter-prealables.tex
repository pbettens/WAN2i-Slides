\section{Préalables}

\begin{frame}[fragile]
  \frametitle{Préalables}
\begin{center}
	\Huge{\bf\color{blue}Préalables}
\end{center}
\begin{flushright}
  \item Définition
  \item Utilité
\end{flushright}
\end{frame}

\begin{frame}[fragile]
  \frametitle{Préalables}
\textbf{WAN \textit{wide area network}}
\begin{itemize}
	\item Un WAN peut être vu comme une interconnexion de LAN
	\item Ce qui caractérise un WAN, c'est la \textbf{distance} entre les hôtes
	... mais pas seulement
	\begin{itemize}
		\item ce sont des \textbf{personnes différentes} qui gèrent les composants
		\item les routeurs connectent des réseaux de \textbf{technologies différentes}
	\end{itemize}

	\item Deux types de composants
	\begin{itemize}
		\item les lignes de transmission (fil de cuivre, fibre optique, liaison radio, ...)
		\item les équipements de commutation (routers)
	\end{itemize}
\end{itemize}
\end{frame}

\begin{frame}[fragile]
  \frametitle{Préalables}
\textbf{Utilité} \\
Une entreprise veut connecter plusieurs filiales, géographiquement distantes, entre elles
\begin{itemize}
	\item Par le biais de lignes louées
	\item \textit{via} Internet, par le biais de VPN (\textit{virtual private network})
\end{itemize}

Exemples
\begin{itemize}
	\item Internet
	\item Le téléphone
	\item Les réseaux d'entreprise
\end{itemize}
\end{frame}
